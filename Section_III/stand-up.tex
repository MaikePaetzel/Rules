\clearpage
\sffamily
{\bfseries \color[rgb]{0.4,0.4,0.4} Part F: Stand Up}
\phantomsection
\addcontentsline{toc}{subsection}{Part F: Stand Up}


\bigskip

The goal of the stand up challenge is to have a robot stand upright on their feet, without human assistance, from a lay down position, according to the rules.

\bigskip

{\bfseries Run Setup}

\smallskip

The initial setup of a run is as follows:

\begin{enumerate}

\item The robot is placed on the field, laying down, on the front or on the back as chosen by the team taking the technical challenge.

\item The referee blows the whistle to start the run.

\item Teams may start the robot manually by pressing a button when the run starts. But the robot must not be touched after the referee blew the whistle. 

\item A chronometer is started when the referee blows the whistle.
\end{enumerate}

{\bfseries Run evaluation}

\smallskip

The chronometer is stopped when the run ends. The causes for the end of a run and
the possible results are as following:
\begin{itemize}
\item \textit{Failure}
  \begin{itemize}
    \item The robot is not able to stand up.
     \end{itemize}
\item \textit{Partial success}
  \begin{itemize}
    \item The robot is able to stand on its feet, with only the feet touching the grass, but falls before being able to stand upright.
  \end{itemize}
\item \textit{Success}
  \begin{itemize}
    \item The robot stands up and stays up, without moving, for 5 seconds.
  \end{itemize}
\end{itemize}

{\bfseries Trials and ranking}

\smallskip

A trial is considered as successful if at least 2 runs in a row resulted in \textit{Success}. A trial is considered as
partially successful if at least 2 runs in a row resulted in \textit{Success} or \textit{Partial success}.

The teams are ranked according to the following criteria on their best batch:
\begin{enumerate}
\item \textit{Success}.
\item \textit{Partial success}.
\item Average time for \textit{Successful} runs.
\item Average time for \textit{Partially successful} runs.
\end{enumerate}
