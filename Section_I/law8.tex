\clearpage
\sffamily
{\bfseries\color[rgb]{0.4,0.4,0.4}
Law 8 -- The Start and Restart of Play}
\phantomsection
\addcontentsline{toc}{subsection}{Law 8 -- The Start and Restart of Play}

\bigskip

A kick-off starts both halves of a match, both halves of extra time and restarts
play after a goal has been scored.
Free kicks (direct or indirect), penalty kicks, throw-ins goal kicks and corner
kicks are other restarts (see law 13-17).

\bigskip

If an infringement occurs when the ball is not in play this does not change how
play is restarted.

\bigskip

{\bfseries Definition of kick-off}

\headlinebox

A kick-off is a way of starting or restarting play:

\begin{itemize}
\item at the start of the match 
\item after a goal has been scored 
\item at the start of the second half of the match 
\item at the start of each period of extra time, where applicable
\end{itemize}

A goal may (new:) not be scored directly from the kick-off by the team taking the kick-off.
Either the ball must move entirely outside the centre circle or must be touched
by another player before being kicked into the goal.
If the ball is kicked directly into the goal a goal-kick
is awarded to the opposing team.

\bigskip

{\bfseries Procedure }

\headlinebox

Before a kick-off at the start of the match or extra time 

\begin{itemize}
\item \added{the referee decides randomly which team attacks which goal}\removed{a coin is tossed and the team that wins the toss decides which goal it will attack in the first half of the match.}
\item \added{the referee decides randomly which teams has kick-off in the first half of the match.}\removed{the other team takes the kick-off to start the match.}
\item \added{the team that was not given kick-off in the first half of the match takes the kick-off to start the second half of the match.}\removed{the team that wins the toss takes the kick-off to start the second half of the match.}
\item in the second half of the match, the teams change ends and attack the opposite goals. 
\end{itemize}

\bigskip

{\bfseries Kick-off}

\begin{itemize}
\item after a team scores a goal, the kick-off is taken by the other team. 
\item all players must be in their own half of the field of play 
\item the opponents of the team taking the kick-off are outside the center
      circle until it is in play
      \greyed{(replaces: the opponents of the team taking the kick-off are at
      least 9.15 m (10 yds) from the ball until it is in play)}
\item the ball must be stationary on the centre mark
\item the referee gives a signal
\item the ball is in play when it is kicked and \added{moves at least 5 cm} \removed{clearly moves
      (new: as determined by the referee} or 10 seconds after the referee gave the signal)
\greyed{\item 
(suspended) the kicker must not touch the ball again until it has touched another player }
\end{itemize}

{\bfseries Infringements and sanctions}

\headlinebox

\greyed{(suspended: If the player taking the kick-off touches the ball again before it has touched another player:

\begin{itemize}
\item an indirect free kick is awarded to the opposing team to be taken from the position of the ball when the infringement occurred (see Law 13 -- Position of free kick)
\end{itemize}
}

In the event of any other infringement of the kick-off procedure: 

\begin{itemize}
\item the kick-off is retaken
\end{itemize}


{\bfseries Definition of dropped ball}

\headlinebox

A dropped ball is a method of restarting play when, while the ball is still in play, the referee is required to stop play temporarily for any reason not mentioned elsewhere in the Laws of the Game. 

\bigskip

{\bfseries Procedure}

\headlinebox

The game is continued at the centre mark. A goal can be scored directly from a dropped ball. The procedure for dropped ball is the same as for kick-off, except that the players of both teams must be outside the centre circle. The ball is in play immediately
after the referee gives the signal. If a player moves too close to the ball before the referee gives the signal, a kick-off
is awarded to the opponent team.


\greyed{(replaces:
The referee drops the ball at the place where it was located when play was stopped, unless play was stopped inside the goal area, in which case the referee drops the ball on the goal area line parallel to the goal line at the point nearest to where the ball was located when play was stopped.


\bigskip

Play restarts when the ball touches the ground.)}

\bigskip

{\bfseries \removed{Infringements and sanctions}}

\headlinebox

\removed{The ball is dropped again:}

\begin{itemize}
\item \removed{if it is touched by a player before it makes contact with the ground}
\item \removed{if the ball leaves the field of play after it makes contact with the ground, without a player touching it}
\end{itemize}

\bigskip

\greyed{(suspended:
If the ball enters the goal: 

\begin{itemize}
\item if a dropped ball is kicked directly into the opponents' goal, a goal kick is awarded 
\item if a dropped ball is kicked directly into the team's own goal, a corner kick is awarded to the opposing team
\end{itemize}
}
