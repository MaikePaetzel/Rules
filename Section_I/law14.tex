\clearpage
\sffamily
{\bfseries \color[rgb]{0.4,0.4,0.4}{Law 14 -- The Penalty Kick}}
\phantomsection
\addcontentsline{toc}{subsection}{Law 14 -- The Penalty Kick}

\bigskip
A penalty kick is awarded against a team that commits one of the \removed{six}
\greyed{(replaces: ten}) offences for which a direct free kick is awarded,
inside its own penalty area and while the ball is in play.

\bigskip

A goal may be scored directly from a penalty kick.

\bigskip

\greyed{(suspended: Additional time is allowed for a penalty kick to be taken at the end of each half or at the end of periods of extra time.)}

\bigskip

{\bfseries Position of the ball and the players }

\headlinebox

The ball:

\begin{itemize}
\item must be placed on the penalty mark.
\end{itemize}

\removed{(new:) During penalty shoot-out, the player taking the penalty kick:}

\begin{itemize}
\item \removed{must be properly identified}
\end{itemize}

The defending goalkeeper:

\begin{itemize}
\item must remain on his goal line, \removed{facing the kicker,} between the goalposts until the ball has been kicked 
\end{itemize}

The players other than the kicker must be:

\bigskip
\begin{itemize}
\item inside the field of play
\item \greyed{(suspended: outside the penalty area)}
\item behind the penalty mark
\item at least 0.75m for KidSize and 1.5m for AdultSize from the
      penalty mark \greyed{(replaces: 9.15m)}
\end{itemize}

\bigskip

{\bfseries Procedure}

\headlinebox

If a penalty kick is taken during the normal course of play the same procedure
as in regular direct free kicks is applied.

\bigskip
During penalty shoot-out:
\begin{itemize}
\item After the players have taken positions in accordance with this law,
      the referee signals for the penalty kick to be taken
\item The player taking the penalty kick must kick the ball forward
\item \greyed{(suspended: He must not play the ball again until it has touched another player)}
\item The ball is in play when it is kicked and moves forward \added{for at least 5 cm}
\end{itemize}

\greyed{(replaces:)
When a penalty kick is taken during the normal course of play, or time
has been extended at half-time or full time to allow a penalty kick to
be taken or retaken, a goal is awarded if, before passing between the
goalposts and under the crossbar:

\begin{itemize}
\item the ball touches either or both of the goalposts and/or the crossbar
and/or the goalkeeper 
\end{itemize}}

\bigskip

The trial ends after 60 seconds.
It may be extended until the ball comes to a complete stop if the ball is still
moving at the time the 60 seconds are over.
The trial also ends if the ball stops being entirely inside the goal area or
leaves the field.

\bigskip

\greyed{(replaces:)
The referee decides when a penalty kick has been completed.)}

\bigskip

{\bfseries Infringements and sanctions }

\headlinebox

The same infringements and sanctions as in regular direct free kicks are applied.

\bigskip

\added{If the player taking the penalty kick does not kick the ball forward:}
\begin{itemize}
\item \added{The referee allows the penalty kick to be retaken once. If the ball is moved backwards again, the ball is considered in play and the penalty kick continues.}
\end{itemize}

\bigskip
\greyed{(replaces:)
If the referee gives the signal for a penalty kick to be taken and,
before the ball is in play, one of the following occurs:

the player taking the penalty kick infringes the Laws of the Game:

\begin{itemize}
\item the referee allows the kick to be taken
\item if the ball enters the goal, the kick is retaken
\item if the ball does not enter the goal, the referee stops play and the
match is restarted with an indirect free kick to the defending team
from the place where the infringement occurred
\end{itemize}

\bigskip

the goalkeeper infringes the Laws of the Game:

\begin{itemize}
\item the referee allows the kick to be taken 
\item if the ball enters the goal, a goal is awarded 
\item if the ball does not enter the goal, the kick is retaken 
\end{itemize}

\bigskip

a team-mate of the player taking the kick infringes the Laws of the
Game: 

\begin{itemize}
\item the referee allows the kick to be taken 
\item if the ball enters the goal, the kick is retaken 
\item if the ball does not enter the goal, the referee stops play and the
match is restarted with an indirect free kick to the defending team
from the place where the infringement occurred 
\end{itemize}

\bigskip

a team-mate of the goalkeeper infringes the Laws of the Game: 

\begin{itemize}
\item the referee allows the kick to be taken 
\item if the ball enters the goal, a goal is awarded 
\item if the ball does not enter the goal, the kick is retaken 
\end{itemize}

\bigskip

a player of both the defending team and the attacking team infringe the
Laws of the Game:

\begin{itemize}
\item the kick is retaken
\end{itemize}

\bigskip

If, after the penalty kick has been taken:

the kicker touches the ball again (except with his hands) before it has
touched another player:

\begin{itemize}
\item an indirect free kick is awarded to the opposing team, the kick to be
taken from the place where the infringement occurred (see Law 13 --
Position of Free Kick)
\end{itemize}

\bigskip

the kicker deliberately handles the ball before it has touched another
player:

\begin{itemize}
\item a direct free kick is awarded to the opposing team, to be taken from the
place where the infringement occurred (see Law 13 -- Position of free
kick)
\end{itemize}

\bigskip

the ball is touched by an outside agent as it moves forward:

\begin{itemize}
\item the kick is retaken
\end{itemize}

\bigskip

the ball rebounds into the field of play from the goalkeeper, the
crossbar or the goalposts and is then touched by an outside agent:

\begin{itemize}
\item the referee stops play
\item play is restarted with a dropped ball at the place where it touched the
outside agent, unless it touched the outside agent inside the goal area,
in which case the referee drops the ball on the goal area line parallel
to the goal line at the point nearest to where the ball was located
when play was stopped)
\end{itemize}
}
\color{black}
